\begin{abstract}
Suppose Alice and Bob are communicating in order to compute some function $f$, but instead of a classical communication channel they have a pair of walkie-talkie devices. They can use some classical communication protocol for $f$ where each round one player sends a bit and the other one receives it. The question is whether talking via walkie-talkie gives them more power? Using walkie-talkie instead of a classical communication channel allows players two extra possibilities: to speak simultaneously (but in this case they do not hear each other) and to listen at the same time (but in this case they do not transfer any bits). The motivation for this kind of a communication model is coming from the study of the KRW conjecture. We show that for some definitions this non-classical communication model is, in fact, more powerful than the classical one as it allows to compute some functions in a smaller number of rounds. We also prove lower bounds for these models using both combinatorial and information theoretic methods.
\end{abstract}


\section{Introduction}
In the classical communication complexity introduced by Yao \cite{Yao79}  two players, Alice and Bob,  are trying to compute $f(x,y)$, for some function $f$, where Alice knows only $x$ and Bob knows only $y$. Alice and Bob can communicate by sending bits to each other, one bit
per round. The essential property of this \emph{classical} model is that in every round of communication one player
sends some bit and the other one receives it.

We define three new communication models that generalize the classical one and resemble
communication over so-called \emph{half-duplex channels}.  A well-known example of half-duplex
communication is talking via walkie-talkie: one has to hold a ``push-to-talk'' button to speak to another person, and one has to release it w want to listen. 
If by accident two persons try to speak simultaneously then they do not hear each other. 
We consider communication models where players are allowed to speak simultaneously. 
Every round each player chooses one of three
actions: send $0$, send $1$, or receive. There are three different types of rounds.
If one player sends some bit and the other one receives then
communication works like in the classical case, we call such rounds \emph{normal}. 
If both players send bits in during the round then these bits get lost (the same happens if two persons try
to speak via walkie-talkie simultaneously), we call these rounds \emph{spent}. 
If both players receive, we call these rounds \emph{silent}. We distinguish three
possible models, based on what happens in silent rounds. If in silent rounds
both players receive $0$, i.e., players cannot distinguish a silent round from a normal round
where the other player sends $0$, we call this model \emph{half-duplex communication with zero}. A somewhat similar model was studied in~\cite{EKS17} for multi-party communication with the noisy broadcast channel. Two other models, we will define later.

In this paper, we study communication complexity of Boolean functions that are hard in the classical case. It is important to note that we care about multiplicative constants. Every classical communication can be viewed as half-duplex communication with zero and every half-duplex communication with zero can be simulated with classical communication doubling the number of rounds (see Theorem~\ref{thm:trivial-ub} and~\ref{thm:trivial-lb-2}). 
So the complexity of half-duplex communication is sandwiched between the complexity of the classical case and a half of it. The task of this study is to improve these bounds.

\subsection{Motivation}
The original motivation to study these kinds of communication models arose from the question of the
complexity of Karchmer-Wigderson games~\cite{KW88} for multiplexers. The \emph{Karchmer-Wigderson game
for a function} $f:\{0,1\}^n\to \{0,1\}$ (\emph{KW game}) is a (classical) communication problem where Alice is given
$x\in f^{-1}(0)$, Bob is given $y\in f^{-1}(1)$, and they want to find $i\in[n]$ such that $x_i\neq
y_i$. Let $D(KW(f))$ be a minimal number of rounds that is enough to solve KW game for $f$ on any pair of possible inputs.
\begin{conjecture}[KRW conjecture \cite{KRW95}]
Let $f:\{0,1\}^n\to\{0,1\}$ and $g:\{0,1\}^m\to\{0,1\}$ be Boolean non-constant functions.
Then $D(KW(g\circ f))\approx D(KW(g)) + D(KW(f))$, where $g\circ f$ denotes a \emph{composition} $g\circ f : (\{0,1\}^n)^m\to\{0,1\}$ is defined by
\(
    (g\circ f)(x_1,\dotsc,x_m) = g(f(x_1),\dotsc, f(x_m))
\)
where $x_1,\dotsc,x_n\in\{0,1\}^m$.
\end{conjecture}
This conjecture implies super-logarithmic formula depth lower bound (and hence super-polynomial size lower bound): we can start with function on $\log n$ variables
that requires logarithmic depth and construct a formula on $n$ variables that requires
super-logarithmic depth.
In attempt to prove it a lot of work has been done studying KW games where one or both functions are replaced with \emph{universal relations}~\cite{HW90, EIRS01, GMWW17}. A \emph{multiplexer} (or \emph{indexing function}) is a function $M_n:\{0,1\}^{2^n}\times
\{0,1\}^n\to\{0,1\}$, such that $M_n(t,i) = t[i]$, i.e., $M_n$ interprets the first part of its input
as the truth table of some function $f:\{0,1\}^n\to\{0,1\}$ and the second part as an input $x$ to the function, and outputs $f(x)$. 
Multiplexers are similar to universal relations in the sense that there is a natural reduction 
from a KW game for some function $f:\{0,1\}^n\to\{0,1\}$ to a KW game for multiplexer $M_n$: if Alice and Bob are given $x$ and $y$ in the game for $f$ we give them $(tt(f), x)$ 
and $(tt(f),y)$, respectively, in the game for $M_n$, where $tt(f)$ is a truth table of function $f$. 
On the other hand, multiplexers are functions, not relations, so proving analogous
results for multiplexers would be one step toward proving KRW conjecture.
Unfortunately, all the techniques that were used for universal relations cannot be applied directly 
to multiplexers because it is impossible to give Alice and Bob the same input string
(all these techniques exploited the symmetry of universal relations that allows giving players the same input string, but this is impossible for functions because inputs of Alice and Bob come from disjoint sets).

Suppose now that Alice and Bob are playing KW game for multiplexer $M_n$: Alice is
given $(tt(f), x)$, $x\in f^{-1}(0)$, and Bob is given $(tt(g), y)$, $y\in g^{-1}(1)$. If the
players are also given a promise that $f=g$ (note that $f$ and $g$ are parts players inputs, so Alice and Bob plays KW game for $M_n$ on a subset of inputs) then they can use a protocol for KW 
game for $f$. 
However, what if they do not have such a promise (i.e., all inputs are possible, in particular, such that $f\neq g$)?
Alice can still try to act as if she plays KW game for $f$, 
Bob at the same time can try to act as if he plays KW game for $g$,
but if in fact $f\neq g$ then in some round of this ``mixed'' protocol they might
both want to send or both want to receive at the same time.  
Such protocol ``mixing'' is impossible in the classical model. 
To make it possible we extend the communication model by allowing players to speak or listen simultaneously. How does it affect the communication complexity? Answering this question we care about multiplicative constants~--- 
if in this model all (hard) functions become two times easier in respect to the classical case then this model is useless for proving KRW conjecture.
As a first step toward answering this question, we study half-duplex communication complexity of Boolean functions $\{0,1\}^n\times\{0,1\}^n\to \{0,1\}$ in respect to the classical case. 

\subsection{Organization of this paper}
In Section~\ref{sec:defs}, we give definitions for considered communication models. Then, in Section~\ref{sec:trivial-bounds}, we prove trivial upper and lower bounds that follows immediately from the definitions. Next, in Section~\ref{sec:methods}, we discuss methods for proving communication complexity lower bounds. In Sections~\ref{sec:hds}, \ref{sec:hdz} and \ref{sec:hda}, we present our main results, upper and lower bounds for proposed communication models. Finally, in Section~\ref{sec:open-questions}, we state several open questions.

\section{Definitions}\label{sec:defs}
\begin{definition}
Let $X$, $Y$, and $Z$ be some finite sets. We say that two players, Alice and Bob, are solving
\emph{half-duplex communication problem} for relation $R \subseteq X\times Y\times Z$ if sets $X$, $Y$, $Z$, and
relation $R$ are known by both players, Alice is given some $x\in X$, Bob is given some $y\in Y$,
and players want to find some $z\in Z$ such that $(x,y,z)\in R$, by communicating to each other. 
The communication is organized into rounds. At every round, each player decides 
(depending only on their inputs and previous communication) to
do one of three available actions: send 0, send 1 or receive.
If one player sends some bit $b\in\{0,1\}$ and the other one receives then the latter gets bit $b$, we
call such rounds \emph{normal}.
If both players send bits at the same time then these bits get lost, we call such rounds
\emph{spent} (it is crucial that the player that is sending cannot distinguish whether this round is normal or spent).
If both players receive at the same time, we call such rounds \emph{silent}.
There are three variants of half-duplex communication problem depending
on how silent rounds work.
\begin{itemize}
\item In a silent round both players receive \emph{nothing}, so it is possible for both players to distinguish
    a silent round from a normal one, the corresponding problem is called \emph{half-duplex communication problem with silence}.

\item In a silent round both players receive $0$, i.e., players cannot distinguish a silent round from a normal round where the other player sends $0$, the corresponding problem is called 
\emph{half-duplex communication problem with zero};

\item In a silent round each player receives some arbitrary bit, not necessarily the same as the other player; the corresponding problem is called \emph{half-duplex communication problem with adversary}.
\end{itemize}
We say that half-duplex communication problem for $R$ is \emph{solved} if at the end of communication both players know some $z$, such that $(x,y,z)\in R$.
\end{definition} 

Next, we define a notion of \emph{communication protocol}. 
In the classical case, a protocol is a binary rooted tree that describes communication of players on all possible inputs: every internal node corresponds to a state of communication and defines which of players is sending
this round. Unlike the classical case in half-duplex
communication player does not always know what the
other's player action was ~--- the information about it can be ``lost,'' i.e., in spent rounds player do not know what the other player's action was. 
It means that a player might not know what node of the protocol corresponds
to the current state of communication. 
Note also that solving half-duplex communication problem with zero 
there is no need to send zeros~--- player can receive instead and 
the other player will not notice the difference.
Keeping all this in mind, we give the following definition 
of half-duplex protocol.
\begin{definition}\label{ref:hds-protocol}
\emph{Half-duplex communication protocol with silence} that solves a relation $R\subset X\times Y \times Z$ is a pair 
$(T_A, T_B)$ of rooted trees that describe how Alice and Bob communicate
on all possible inputs $(x,y)\in X\times Y$.
Every node of $T_A$ corresponds to a state of Alice, every node of $T_B$ 
to a state of Bob.  Every leaf $l$ is labeled with $z_l\in Z$. 
Let $\cA = \{\texttt{send(0)}, \texttt{send(1)}, \texttt{receive}\}$ be the set of possible actions, and $\cE = \{\texttt{send(0)}, \texttt{send(1)},
\texttt{receive(0)}, \texttt{receive(1)}, \texttt{silence}\}$ 
be the set of all possible events.  Every node $v$ of $T_A$ and 
(of $T_B$) is labeled with two functions $g_v: X\to \cA$ 
($g_v: Y\to \cA$) and $h_v: \cE\to C(v)$, where $C(v)$ 
is a set of child nodes of $v$.  Root nodes of $T_A$ and
$T_B$ correspond, respectively, to the initial states of Alice and Bob.
If Alice (Bob) is in a state that corresponds to node $v\in T_A$ ($v\in T_B$), 
then she does action $g_v(x)$ (he does action $g_v(y)$). 
Events of both players are defined in a natural way by their actions in this round. The next node of the protocol is defined by the function $h$.
When players reach a leaf they stop (they always reach a leaf simultaneously).  The protocol is correct if for every input pair 
$(x,y)\in X\times Y$ communication ends 
in a pair of leaves labeled with the same 
$z\in Z$ such that $(x,y,z) \in R$.

\emph{Half-duplex communication protocol with zero} is defined in the 
same way with a different set of possible events 
$\cE = \{\texttt{send(1)}, \texttt{receive(0)}, \texttt{receive(1)}\}$,
i.e it does not include send(0).

\emph{Half-duplex communication protocol with adversary} 
that solves a relation $R\subset X\times Y \times Z$ is a pair 
$(T_A, T_B)$ of rooted trees that describe how Alice and Bob communicate
on all possible inputs $(x,y)\in X\times Y$ and for any strategy of 
adversary $w\in\{0,1\}^*$. The structure of the protocol is the same
as in half-duplex communication protocol with zero, but with 
$\cE = \{\texttt{send(0)}, \texttt{send(1)}, \texttt{receive(0)}, \texttt{receive(1)}\}$. 
If both players decide to receive in round $i$, then Alice and Bob receive
bits $w_{2i-1}$ and $w_{2i}$ respectively.
The protocol is correct if for every input pair 
$(x,y)\in X\times Y$ and any strategy of 
adversary $w\in\{0,1\}^*$ communication ends
in two leaves labeled with the same $z\in Z$ such that $(x,y,z) \in R$.

For each of these models, a \emph{partial transcript after $k$ rounds} is a pair $(\pi_a,\pi_b)$ of length-$k$ sequences over $\cE$ that lists the events observed by Alice and Bob, respectively, after running some protocol on a pair of inputs for $k$ rounds.
\end{definition}
The cardinality of set $\cE$ upper bounds arity of trees $T_A$ and $T_B$: arity is $5$ for half-duplex communication with silence,
$3$ for half-duplex communication with zero, and $4$ for half-duplex communication with the adversary.

\begin{definition}
Half-duplex communication protocol \emph{solves} a communication 
problem for function $f:X\times Y\to Z$ 
if it solves a relation $R(f) = \{(x,y,f(x,y))\mid x\in X, y\in Y\}$.
\end{definition}

The classical communication complexity of a communication problem for function $f$, $D(f)$, is defined
in terms of the minimal depth of a protocol solving it. Analogously, we define
communication complexity for half-duplex communication problems.
\begin{definition}
The minimal depth of a communication protocol solving half-duplex communication problem for function
$f$ with silence, with zero, with adversary, define \emph{half-duplex communication complexity of function}
$f$ with silence, denoted $D^{hd}_s(f)$, with zero, denoted $D^{hd}_0(f)$, with adversary, denoted
$D^{hd}_a(f)$, respectively. Analogously, we define \emph{half-duplex communication complexity of relation} $R$ with silence, $D^{hd}_s(R)$, with zero, $D^{hd}_0(R)$, and with adversary, $D^{hd}_a(R)$.
\end{definition}

In this paper we study half-duplex communication complexity for a special case of Boolean 
functions $\{0,1\}^n\times\{0,1\}^n\to \{0,1\}$ (i.e., $X=Y=\{0,1\}^n$, $Z=\{0,1\}$).
\begin{definition}\mbox{}
\begin{itemize}
    \item \emph{Equality function$  $} $\EQ_n: \{0,1\}^n\times\{0,1\}^n\to \{0,1\}$, such that $\EQ_n(x,y) = 1 \iff x=y$.
    \item \emph{Inner product function} $\IP_n:\{0,1\}^n\times\{0,1\}^n\to \{0,1\}$, such that $\IP_n(x,y) = \bigoplus_{i\in[n]} x_i y_i$.
    \item \emph{Disjointness function} $\DISJ_n:\{0,1\}^n\times\{0,1\}^n\to \{0,1\}$, such that 
    $\DISJ_n(x,y) = 1 \iff \forall i: x_i \neq 1 \lor y_i \neq 1$.
\end{itemize}
All these function require $n$ bits of communication in the classical model.
\end{definition}

\section{Trivial bounds}\label{sec:trivial-bounds}
As far as half-duplex communication generalizes classical 
communication the following upper bound is immediate.
\begin{theorem}\label{thm:trivial-ub}
For every function $f:\{0,1\}^n\times \{0,1\}^n \to \{0,1\}$,
$D^{hd}_s(f)\le D^{hd}_0(f)\le D^{hd}_a(f)\le D(f).$
\end{theorem}
\begin{proof}
Every classical communication protocol can be embedded in half-duplex communication protocol
that does not use spent and silent rounds.
\end{proof}

Next theorem shows that one can always transform half-duplex protocol with zero or with the adversary into
 a classical communication protocol of double depth. 
\begin{theorem}\label{thm:trivial-lb-2}
For every function $f:\{0,1\}^n\times \{0,1\}^n \to \{0,1\}$, 
$\frac{D(f)}{2}\le D^{hd}_0(f) \le D^{hd}_a(f).$
\end{theorem}
\begin{proof}
Every $t$-round half-duplex communication protocol with zero or with the adversary can be transformed into $2t$-round classical communication protocol. Every round of the original protocol corresponds
to two consecutive rounds of the new one: on the first round Alice sends a bit she was sending in the original protocol or sends $0$ if she was receiving, at second round Bob does the same thing.
\end{proof}

As we will see later, half-duplex protocols with silence can use silent rounds as an additional third 
symbol and hence not every $t$-round half-duplex protocol with silence can be embedded in $2t$ 
classical protocol. The following theorem shows that instead, we can embed every such protocol in a classical protocol with $3t$ rounds.
\begin{theorem}\label{thm:trivial-lb-3}
For every function $f:\{0,1\}^n\times \{0,1\}^n \to \{0,1\}$,
$D^{hd}_s(f) \ge \frac{D(f)}{3}.$
\end{theorem}
\begin{proof}
Every $t$-round half-duplex communication protocol with silence can be transformed into $3t$-round
classical communication protocol. Every round of the original protocol corresponds to three consecutive
rounds of the new one: on the first round, Alice sends $1$ to indicate if she was sending a bit in the original protocol, or sends $0$ otherwise, at second round Bob does the same thing symmetrically. After that, they are both aware of the intentions of each other. If they were both planning to send, they could skip the third round. If they were both planning to receive, then they can assume that they heard silence. If one player was planning to send and the other one was planning to receive they can perform such action on the third round.
\end{proof}
\begin{remark}
    Theorems~\ref{thm:trivial-ub}, \ref{thm:trivial-lb-2}, and \ref{thm:trivial-lb-3} holds also for 
    $f:\{0,1\}^n\times\{0,1\}^n\to \{0,1\}^k$.
\end{remark}


\section{Methods for lower bounds}\label{sec:methods}
\subsection{Rectangles}\label{subsec:rectangles}
Many lower bounds on classical communication complexity were proved by considering
combinatorial rectangles associated with the nodes of communication protocol~\cite{KN97}: 
it is easy to see that every node $v$ of the (classical)
protocol corresponds to a combinatorial rectangle $R_v = X_v\times Y_v$, 
where $X_v\subseteq X$, $Y_v\subseteq Y$, such that if Alice and Bob are given an input from $R_v$
then their communication will necessarily pass through node $v$. This implies that the rectangles associated with the child nodes of $v$ define a subdivision of $R_v$.

There is a general technique~\cite{KN97} for proving lower bounds using associated combinational rectangles in: if for some sub-additive measure  $\mu$ defined on combinatorial rectangles we show both a lower bound on the measure of $X\times Y$, the rectangle in the root node, i.e., $\mu(X\times Y)\ge \mu_r$ for some $\mu_r>0$, and
an upper bound on the measure of rectangles in leaves, i.e., for every leaf $l$ the measure
of the corresponding rectangle $R_l$ is at most $\mu_\ell$ for some $\mu_\ell>0$, 
then we can claim lower bound of $\log_2(\mu_r/\mu_\ell)$ on the depth of the protocol.

One of the most studied sub-additive measure on rectangles is $\mu_M(R)$ that is 
equal to the minimal number of \emph{monochromatic} 
rectangles that covers $R$. Rectangle $R$ is \emph{$z$-monochromatic} in respect to function $f$ 
for some $z\in Z$ if for all $(x,y)\in R$, $f(x,y) = z$.
As far as both players have to come up with the same answer at the end of 
communication every rectangle in leaves is monochromatic, thus for this measure $\mu_\ell = 1$.

We can use almost the same technique for half-duplex protocols. 
There are some technical differences that we have to keep in mind. 
First of all, we can apply this idea to both trees $T_A$ and $T_B$. 
We should also note that trees $T_A$ and $T_B$ are non-binary; hence
arity became the base of the logarithm. 
Secondly, we should be careful while defining associated 
combinatorial rectangles for half-duplex protocols with adversary~--- 
in case of silent rounds the next node of the protocol depends also 
on a strategy $w$ of adversary, so we have to formally
consider $w$ it as a part of input. This leads to the following lower 
bound for equality.
\begin{theorem}\label{thm:private-arity-bounds-eq}
\mbox{}
\begin{itemize}
\item $D^{hd}_s(\EQ_n) \ge \log_5{2^n} = n/\log 5$,

\item $D^{hd}_0(\EQ_n) \ge \log_3{2^n} = n/\log 3$,

\item $D^{hd}_a(\EQ_n) \ge \log_{4}{2^n} = n/2$.
\end{itemize}
\end{theorem}
\begin{proof}
Let $\mu = \mu_M$. All leaf rectangles are monochromatic, $\mu_\ell = 1$. Every $1$-monochromatic rectangle is of size one: if some rectangle contains two elements, say $(x,x)$ and $(x',x')$, then it also contains $(x,x')$ and $(x',x)$, so it is not $1$-monochromatic. Thus, the root rectangle has measure at least $\mu_r = 2^n + 1$ (see \cite{KN97} for more information). 
\end{proof}
Surprisingly, as we will see later, first two result are sharp up to additive logarithmic term.
We developed an extension of this technique that we call \emph{round elimination}.

\subsection{Round elimination}
Let us fix a protocol for some half-duplex communication problem and consider the first round. Let $R_c = X\times Y$ be the corresponding rectangle of all possible inputs.
We can subdivide $R_c$ in nine rectangles, one for each possible combination of actions.
\begin{center}
\begin{tabular}{c|c|c|c}
\bf Alice\textbackslash Bob &  \tt send(0) & \tt send(1) &  \tt receive \\\hline 
\tt send(0)  & $R_{00}$ & $R_{01}$ & $R_{0r}$ \\\hline
\tt send(1)  & $R_{10}$ & $R_{11}$ & $R_{1r}$ \\\hline
\tt receive  & $R_{r0}$ & $R_{r1}$ & $R_{rr}$ 
\end{tabular}
\end{center}
Consider two rectangles: $R_{good} = R_{00} \cup R_{01} \cup R_{0r}$ and $R_{bad} = R_{0r}\cup
R_{1r}$. If we restrict $f$ to be a partial function defined 
only on $R_{good}$, i.e., players will always get some 
$(x,y)\in R_{good}$, then there is no need in the first round~---
the information the players get about the other part of the input is fixed: Alice does not get any
information, Bob can receive $0$ if he decide to receive. On the other hand if we restrict $f$ to $R_{bad}$ then the first round is still needed: Bob can receive both $0$ and $1$ and
this information in necessary to proceed to the next round. Lets call a rectangle $R$ \emph{good for (partial) function $f$} if restricting $f$ to $R$ makes the first round unnecessary (i.e., protocol without the first round is correct for all $(x,y)\in R$).
The idea of this method is to consider some covering of $R_c$ with a set of \emph{good} rectangles and prove that there is always a good rectangle of large enough measure.  If we can show
that there is always a rectangle of measure at least $\alpha\cdot\mu(R_c)$ then 
we can iterate this idea and claim that protocol depth is at least 
$\log_{1/\alpha}(\mu_r/\mu_\ell)$, where $\mu_r$ is a lower bound 
on the measure of the root rectangle and $\mu_\ell$ is an upper bound 
on the measure of leaf rectangles.

\begin{lemma}\label{lm:rect-elim}
Let $\mu$ be some sub-additive measure on rectangles such that $\mu(X\times Y) \ge \mu_r$ and for
any leaf rectangle $R_l$, $\mu(R_l)\le \mu_\ell$. If for any rectangle $R$ 
there is always a good subrectangle for function $f\restriction R$ of measure at least $\alpha\cdot \mu(R)$ then the depth of the protocol is at least $\log_{1/\alpha} \frac{\mu_r}{\mu_\ell}$.
\end{lemma}
\begin{proof}
We start with $R = X\times Y$. Every round we show that $f \restriction R$ can be restricted 
to some good $R_{good} \subset R$ such that
$\mu(R_{good})\ge \alpha\cdot \mu(R)$, let $R$ to be $R_{good}$, and proceed to the next round
until we reach a leaf. Thus there are at least $\log_{1/\alpha}(\mu_r/\mu_\ell)$ rounds.
\end{proof}

\subsection{Upper bound on internal information}\label{subsec:information}
Another useful tool for proving lower bounds on the communication complexity of problems in the classical model is the upper bound on the information Alice and Bob have learned about the other's inputs, as a function of the number of rounds. 
\begin{definition}
    Let $f$ be a partial function and $\cP$ a half-duplex communication protocol computing $f$, and $\mathcal{D}$ an arbitrary distribution over the range of $f$.
    Let $\cX$, and $\cY$ be the marginal distributions over inputs to Alice and Bob,
    also, let $\Pi_A$ and $\Pi_B$ be the marginal distributions over Alice and Bob's transcripts 
    induced by $\mathcal{D}$.
    An \emph{internal information cost of protocol $\cP$} is
    \(
    \IC_\mathcal{D}(\cP) = I(\cX:\Pi_B\mid\cY) + I(\cY:\Pi_A\mid\cX).
    \)
    For any $k$ let $\Pi_A^k$ and $\Pi_B^k$ be the marginal distributions over Alice and Bob's partial transcripts after running $\cP$ for $k$ rounds induced by $\mathcal{D}$.
    An \emph{internal information cost of first $k$ rounds of $\cP$} is
    \(
    \IC_\mathcal{D}^k(\cP) = I(\cX:\Pi_B^k\mid\cY) + I(\cY:\Pi_A^k\mid\cX).
    \)
\end{definition}

For more information on information theory, we refer to~\cite{CTJ2006,GMWW17}. We use this approach
to prove lower bounds on the inner product using the following Lemma.

\begin{lemma}\label{lm:ip-information-common}
    Let $\mathcal{D}$ be uniform distribution over all input pairs of $\IP_n$ (pairs of $n$-bit strings).
    If any half-duplex communication protocol with silence/zero/adversary computing $\IP_n$ and for every $k$,
    \(
    \IC_\mathcal{D}^k(\cP) \le \alpha k,
    \)
    for some $\alpha\ge 1$, then half-duplex complexity of $\IP_n$ with silence/zero/adversary is at least $n/\alpha$. 
\end{lemma}
To prove this Lemma we use the following property of $\IP_n$ (the proof is given in Appendix).
\begin{lemma}\label{lm:ip-2n}
    Every leaf rectangle of a protocol for $\IP_n$ has size at most $2^n$. 
\end{lemma}
\begin{proof}[Proof of Lemma~\ref{lm:ip-information-common}]
For uniform distribution over all input pairs $H(\cX\mid\cY)+H(\cY\mid\cX)=2n.$
By Lemma~\ref{lm:ip-2n} each leaf of any correct protocol contains at most $2^n$ input pairs in its rectangle, thus $H(\cX\mid\cY,\Pi_B)+H(\cY\mid\cX,\Pi_A)\leq n.$ If $\IP_n$ has a protocol of depth $k$ then
\begin{align*}
\alpha k &\ge I(\cX:\Pi_B^k\mid\cY) + I(\cY:\Pi_A^k\mid\cX) \\
&= 
H(\cX\mid\cY) - H(\cX\mid\cY,\Pi_B^k) + H(\cY\mid\cX) - H(\cY\mid\cX,\Pi_A^k) \ge n.
\end{align*}
\end{proof}

\section{Half-duplex communication with silence}\label{sec:hds}
The main advantage of this model over the other models we consider is that whenever players have
silent round, they learn about it. In some sense they have a third symbol in the alphabet~--- receiving player can get either $0/1$ or a special symbol corresponding to ``silence''.
Next theorem shows how players can take
the advantage of silence to transfer data.

\begin{theorem}
For every $f: \{0,1\}^n\times\{0,1\}^n\to \{0,1\}$, $D^{hd}_s(f)\le \lceil n / \log 3 \rceil + 1$.
\end{theorem}
\begin{proof}
Alice encodes $x$ in ternary alphabet $\{0,1,2\}$
and sends it to Bob: in order to send $0$ or $1$ Alice sends the corresponding bit, sending $2$ is
emulated by receiving (keeping silence). This requires $\lceil\log_3 2^n\rceil = \lceil n /
\log 3\rceil$ bits.  At the last round Bob computes $f(x,y)$ and sends the result back to Alice.
\end{proof}

Using the idea of non-binary encoding, we prove a better upper bound for equality.

\begin{theorem}
$D^{hd}_s(\EQ_n) \le \lceil n/\log 5\rceil + \lceil \log n / \log 3 \rceil + 2$.
\end{theorem}
\begin{proof}
Alice and Bob encode their inputs in alphabet of size five $\{0,1,2,3,4\}$. 
Then they process their inputs symbol by symbol sequentially in $\lceil n/\log 5\rceil$ rounds. 
At round $i$ they process $i$th symbol in the following manner.
\begin{center}
\begin{tabular}{c|c|c}
\bf Symbol & \bf Alice & \bf Bob \\\hline 
0 & \tt send(0)   & \tt receive  \\\hline
1 & \tt send(1)   & \tt receive  \\\hline
2 & \tt receive   & \tt send(0)  \\\hline
3 & \tt receive   & \tt send(1)  \\\hline
4 & \tt receive   & \tt receive
\end{tabular}
\end{center}
If $i$th round is normal then one player can check whether $i$th symbols are different.
If $i$th round is silent then again one player knows if $i$th symbols are different.
If after $\lceil n/\log 5\rceil$ rounds one of the players has already learned that the answer is $0$,
then he or she sends $0$. If this round is not silent, then both players know that the answer is $0$.
Otherwise, Alice and Bob have to make sure that there were no spent rounds.
To check it, Alice sends the number normal rounds 
she was receiving in encoded in ternary, that requires $\lceil \log n / \log 3 \rceil$ rounds.
Bob checks whether this number is equal to the number of rounds he was sending in. If so, inputs are equal.
In the last round, Bob sends the answer back to Alice.
\end{proof}
Using almost the same ideas we can show an upper bound for disjointness.
\begin{theorem}\label{thm:hds-disj}
$D^{hd}_s (\DISJ_n) \le \lceil n/2\rceil + 2$.
\end{theorem}

To prove lower bounds one can use round elimination and get the following lower bound for the inner product (the proof is given in Appendix).

\begin{theorem}\label{thm:hds-ip}
$D^{hd}_s(\IP_n)\ge n/2.$
\end{theorem}

This lower bound can be improved using upper bound on internal information.

\begin{theorem}\label{thm:hds-ip-information}
$D^{hd}_s(\IP_n)\ge n/1.67$.
\end{theorem}
\begin{proof}
    To apply Lemma~\ref{lm:ip-information-common} it is enough to show that 
    $I(\cX:\Pi_B^k\mid\cY) + I(\cY:\Pi_A^k\mid\cX)\leq \alpha k,$ where     $\alpha\le 1.67$.
    We will induct on $k$: the number of rounds.
    For $k=0$, there is only one possible partial transcript for either player, the empty transcript, and thus the result is immediate.
    Now suppose that this is true in round $k$.
    Let $\cE^{k+1}_A$ and $\cE^{k+1}_B$ be the marginal distributions over which event each player will observe.
    Note that \begin{align*}
    I(\cX:\Pi^{k+1}_B\mid\cY)&=H(\cX\mid\cY)-H(\cX\mid\cY,\Pi^{k+1}_B)\\
    &=H(\cX\mid\cY)-H(\cX\mid\cY,\Pi^k_B)+H(\cX\mid\cY,\Pi^k_B)-H(\cX\mid\cY,\Pi^k_B,\cE^{k+1}_B)\\
    &=I(\cX:\Pi^k_B\mid\cY)+I(\cX:\cE^{k+1}_B\mid\cY,\Pi^k_B).
    \end{align*}
    Thus, it suffices to show that $I(\cX:\cE^{k+1}_B\mid\cY,\Pi^k_B)+I(\cY:\cE^{k+1}_A\mid\cX,\Pi^k_A)\leq \alpha.$ 
    Note that 
    \[
    I(\cX:\cE^{k+1}_B\mid\cY,\Pi^k_B)
    = H(\cE^{k+1}_B\mid\cY,\Pi^k_B)-H(\cE^{k+1}_B\mid\cY,\Pi^k_B,\cX)
    = H(\cE^{k+1}_B\mid\cY,\Pi^k_B).
    \]
    The second term here is zero because values of $\cX$ and $\cY$ unambiguously determine the entire protocol. 
    So it is enough to bound $H(\cE^{k+1}_B\mid\cY,\Pi^k_B) =\E_{y,\pi}[H(\cE^{k+1}_B\mid\cY=y,\Pi^k_B=\pi)].$

    Let $\cA^{k+1}_A$ and $\cA^{k+1}_B$ be the marginal distributions over players actions in round $k$. 
    Note that $\cA^{k+1}_B$ is a function of $y$ and $\pi$. 
    If for some pair $(y,\pi)$ Bob sends, 
    i.e. $\cA^{k+1}_B = \texttt{send(0)}$ or $\cA^{k+1}_B = \texttt{send(1)}$, then $H(\cE^{k+1}_B\mid\cY=y,\Pi^k_B=\pi) = 0$. For the sake of brevity we denote $E_{y,\pi}$ an event ``$\cY=y,\Pi^k_B=\pi$'' and $r$ an action $\texttt{receive} \in\cA$.
    \[
    H(\cE^{k+1}_B\mid\cY,\Pi^k_B)
    =\sum_{\substack{(y,\pi):\\\cA^{k+1}_B = r}}\Pr[E_{y,\pi}]\cdot H(\cE^{k+1}_B\mid E_{y,\pi}).
    \]
    Let $\cE_r = \{\texttt{receive(0)}, \texttt{receive(1)}, \texttt{silence}\}$ be a set of events that can happen to a player while receiving. For a pair $(y,\pi)$ such that  $\cA_B^{k+1} = r$,
    \[
    H(\cE^{k+1}_B\mid E_{y,\pi})=\sum_{e\in \cE_r} \Pr[\cE^{k+1}_B = e \mid E_{y,\pi}]\cdot \log\frac{1}{\Pr[\cE^{k+1}_B = e \mid E_{y,\pi}]},
    \]
    If Bob receives then his event in this round is defined by action of Alice.    
    \begin{align*}
    H(\cE^{k+1}_B\mid\cY,\Pi^k_B)
    &=\sum_{\substack{(y,\pi):\\\cA^{k+1}_B = r}}\Pr[E_{y,\pi}]\cdot
    \sum_{e\in \cE_r} \Pr[\cE^{k+1}_B = e \mid E_{y,\pi}]\cdot \log\frac{1}{\Pr[\cE^{k+1}_B = e \mid E_{y,\pi}]}\\
    &=\sum_{e\in \cE_r} \sum_{\substack{(y,\pi):\\\cA^{k+1}_B = r}}
     \Pr[\cE^{k+1}_B = e , E_{y,\pi}]\cdot \log\frac{1}{\Pr[\cE^{k+1}_B = e \mid E_{y,\pi}]}\\
     &=\sum_{a\in \cA} \sum_{\substack{(y,\pi):\\\cA^{k+1}_B = r}}
     \Pr[\cA^{k+1}_A = a , E_{y,\pi}]\cdot \log\frac{1}{\Pr[\cA^{k+1}_A = a \mid E_{y,\pi}]}.
    \end{align*}
    Now we use Jensen's inequality.
    \begin{align*}
    &H(\cE^{k+1}_B\mid\cY,\Pi^k_B)\\
    &\le \sum_{a\in \cA} 
    \Pr[\cA^{k+1}_A = a , \cA^{k+1}_B = r]\cdot \log \sum_{\substack{(y,\pi):\\\cA^{k+1}_B = r}}
    \frac{\Pr[\cA^{k+1}_A = a , E_{y,\pi}]}{\Pr[\cA^{k+1}_A = a , \cA^{k+1}_B = r]}\cdot \frac{1}{\Pr[\cA^{k+1}_A = a \mid E_{y,\pi}]}\\
    &= \sum_{a\in \cA} 
    \Pr[\cA^{k+1}_A = a , \cA^{k+1}_B = r]\cdot \log \sum_{\substack{(y,\pi):\\\cA^{k+1}_B = r}}
    \frac{\Pr[E_{y,\pi}]}{\Pr[\cA^{k+1}_A = a , \cA^{k+1}_B = r]}\\
    &= \sum_{a\in \cA} 
    \Pr[\cA^{k+1}_A = a , \cA^{k+1}_B = r]\cdot \log 
    \frac{\Pr[\cA^{k+1}_B = r]}{\Pr[\cA^{k+1}_A = a , \cA^{k+1}_B = r]}.
    \end{align*}
    Now we use independence of player's action choices.
    \begin{equation*}\label{eq:common-entropy-bound}
    H(\cE^{k+1}_B\mid\cY,\Pi^k_B)
    \le \Pr[\cA^{k+1}_B = r] \cdot \sum_{a\in \cA} 
    \Pr[\cA^{k+1}_A = a] \cdot \log 
    \frac{1}{\Pr[\cA^{k+1}_A = a]}.
    \end{equation*}
    The same argument works for $I(\cY:\Pi_A^k\mid\cX)$ and hence we get,
    \begin{align*}
    I(\cX:\Pi_B^k\mid\cY) + I(\cY:\Pi_A^k\mid\cX)&\leq    
    \Pr[\cA^{k+1}_B = r] \cdot \sum_{a\in \cA} 
    \Pr[\cA^{k+1}_A = a] \cdot \log 
    \frac{1}{\Pr[\cA^{k+1}_A = a]} \\
    &+ 
    \Pr[\cA^{k+1}_A = r] \cdot
    \sum_{a\in \cA} 
    \Pr[\cA^{k+1}_B = a] \cdot \log 
    \frac{1}{\Pr[\cA^{k+1}_B = a]}.
    \end{align*}
    Now let's denote $a_0$ and $a_1$ to be the fractions of inputs for which Alice sends $0$ or $1$, respectively,
    and symmetrically $b_0$ and $b_1$ to be the fractions of inputs for which Bob sends $0$ or $1$, respectively.
    The right hand side of the above inequality can be rewritten as follows.
    \begin{align*}
    &(1 - b_0 - b_1) \cdot \Biggl( a_0\cdot\frac{1}{a_0} + a_1\cdot\frac{1}{a_1} + (1 - a_0 - a_1)\cdot\frac{1}{(1 - a_0 - a_1)} \Biggr)\\
    +\ &(1 - a_0 - a_1) \cdot \Biggl( b_0\cdot\frac{1}{b_0} + a_1\cdot\frac{1}{b_1} + (1 - b_0 - b_1)\cdot\frac{1}{(1 - b_0 - b_1)}\Biggr).
    \end{align*}
    Numerical analysis of this expression shows that it's maximum is less then $1.67$ (for $a_0 = a_1= b_0 = b_1 \approx 0.17$), hence $I(\cX:\Pi_B^k\mid\cY) + I(\cY:\Pi_A^k\mid\cX)\leq 1.67$.
\end{proof}


\section{Half-duplex communication with zero}\label{sec:hdz}
As we have already mentioned before there are only two reasonable actions in this model: send
$1$ or receive. The following theorem shows that half-duplex communication with zero is more powerful than classical communication; namely, it is possible to compute equality
in less than $n$ rounds of communication. 
%This result was unexpected for us~--- our initial
%intuition prompted that in this setting silent and spent rounds would be useless.
\begin{theorem}
$D^{hd}_0(\EQ_n) \le \lceil n/\log 3\rceil + 2\lceil \log n\rceil + 1$.
\end{theorem}
\begin{proof}
Alice and Bob encode their inputs in ternary. 
In the first phase of the protocol, they process their inputs sequentially symbol by symbol in $\lceil n/\log 3\rceil$ rounds. 
At round $i$ they process $i$th symbol in the following manner.
\begin{center}
\begin{tabular}{c|c|c}
\bf Symbol & \bf Alice & \bf Bob \\\hline 
0 & \tt receive  & \tt receive  \\\hline
1 & \tt send(1)  & \tt receive  \\\hline
2 & \tt receive  & \tt send(1) 
\end{tabular}
\end{center}
In the next 2$\lceil \log n\rceil$ they send each other the number of ones 
they sent in the first phase. If inputs were different then one of players must have noticed it.
At the first phase at round $i$ Alice learns if their corresponding symbols 
are $(0, 2)$, $(2,0)$ or $(2,1)$, Bob learns if their symbols are $(0, 1)$ or $(1,0)$.
In the second phase, they can learn whether any of $(1,2)$ situation happened in
the first phase. In the last round, players notify each other if somebody
noticed a mismatch~— in this case the player that noticed it sends $1$.
\end{proof}

The best lower bound for this model is again for $\IP_n$. The next theorem is proved using round elimination (the proof is given in Appendix).
\begin{theorem}\label{thm:hdz-ip}
$D^{hd}_0(\IP_n)\ge n / \log \frac{2}{3 - \sqrt{5}} > n/1.39$.
\end{theorem}
The better lower is proved with information theoretic approach.
\begin{theorem}\label{thm:hdz-ip-information}
$D^{hd}_0(\IP_n)\ge n/1.234$.
\end{theorem}
\begin{proof}
The proof repeats the proof of Theorem~\ref{thm:hds-ip-information}.
The only difference is that in this model players never send 0. So at the end
we end up maximizing 
$
(1 - b_1) \cdot h(a_1) + (1 - a_1) \cdot h(b_1),
$
where $h(p) = p\cdot\log\frac{1}{p} + (1 - p)\cdot\log\frac{1}{1 - p}$ is a binary entropy function. Maximum of this expression is slightly less then $1.234$ ($a_1 = b_1 \approx 0.29$).
\end{proof}

\section{Half-duplex communication with adversary}\label{sec:hda}
The main feature of this model is that receiving player cannot be $100\%$ sure that
the received bit if in fact is ``real'', i.e., this bit originates from the other player,
not from an adversary. The protocol must be correct for any strategy of the adversary.
Our intuition prompts that in this setting silent and spent rounds would be useless.
%There is a common obstacle our combinatorial methods faced when we were trying to prove this conjecture~— 
%it could be the case that players send different bits in spent rounds.
%For some reason, our combinatorial methods do not work 
%in this case which is strange because
%these spend rounds do not transmit any information.
%If we somehow forbid players to send different
%bits in spent rounds (e.g., in this case, we immediately terminate the communication and make players output $0$)
%then we can prove that $\EQ_n$ requires $n$ rounds of communication. 
%The same bound can be achieved if we allow such spent rounds only on distinct inputs. 
Using combinatorial methods, one can show the following two lower bounds (the proofs are given in Appendix).

\begin{theorem}\label{thm:lb-a-eq}
$D^{hd}_a(\EQ_n) \ge n/\log 2.5$.
\end{theorem}
\begin{theorem}\label{thm:lb-a-ip}
$D^{hd}_a(\IP_n)\ge n/\log\frac{7}{3}$.
\end{theorem}
And again better lower bound for $\IP_n$ can be obtained using  information-theoretic approach.
\begin{theorem}\label{thm:hda-ip-information}
$D^{hd}_a(\IP_n) \ge n$.
\end{theorem}
To prove this theorem we use the ideas from the proof of Theorem~\ref{thm:hds-ip-information}: 
in order to apply Lemma~\ref{lm:ip-information-common} we
show that $I(\cX:\Pi_B^k\mid\cY) + I(\cY:\Pi_A^k\mid\cX)\leq k$, and hence we get the desired bound.
The detailed proof is given in Appendix.

Using the same approach we can show $2\log n$ lower bound 
on the complexity of Karchmer-Wigderson relation for parity function.
\begin{definition}
Let $X=f^{-1}(0)$, $Y = f^{-1}(1)$ for some Boolean function $f:\{0,1\}^n\to\{0,1\}$.
The \emph{KW relation for function $f$}, $R_f\subseteq X \times Y \times [n]$, is defined by
$R_f = \{(x,y,i)\mid x_i \neq x_i\}.$
\end{definition}
It it well known that parity function $\oplus_n: \{0,1\}^n\to \{0,1\}$, $\oplus_n(x) = \bigoplus_{i=1}^n x_i$, requires $n^2$ formula size~\cite{Kh72}. 
In the classical case it is equivalent to saying that KW relations for parity requires $2\log n$
rounds of communication. In the proof of Theorem~\ref{thm:hda-ip-information} we shown that
$I(\cX:\cE^{k+1}_B\mid\cY,\Pi^k_B)+I(\cY:\cE^{k+1}_A\mid\cX,\Pi^k_A)\leq 1.$
It allows us to prove the following analogue of this result.

\begin{corollary}
$D^{hd}_a(R_{\oplus_n}) \ge 2\log n$.
\end{corollary}
\begin{proof}
Take the uniform distribution over valid input pairs with a single bit of difference.
Then $H(\cY\mid\cX)+H(\cX\mid\cY)=2\log n$ before any communication takes place. On the other hand it is easy to see that $H(\cY\mid\cX,\Pi_A)+H(\cX\mid\cY,\Pi_B)=0$ at any leaf.
\end{proof}

\section{Open problems}\label{sec:open-questions}
It would be interesting to improve upper and lower bounds for Boolean functions 
for all three half-duplex communication models.
So we propose the following list of open problems.
\begin{enumerate}
\item Prove better upper and lower bounds for the half-duplex model with silence and zero.

\item Is there any $\alpha < 1$ such that for any $f:\{0,1\}^n\times\{0,1\}^n\to\{0,1\}$, $D^{hd}_0(f)\le \alpha n + o(n)$?

\item Is there any $f:\{0,1\}^n\times\{0,1\}^n\to\{0,1\}$, such that at the same time $D(f) \ge n - o(n)$ and $D^{hd}_a(f)\le \alpha n + o(n)$ for some $\alpha<1$.
\end{enumerate}
